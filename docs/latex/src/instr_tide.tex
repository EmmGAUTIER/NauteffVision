
\begin{tikzpicture}[line cap=round, line width=1pt, x=1cm, y=1cm]

  \newcommand{\instrSize}{5}
  \newcommand{\instrCenterX}{0}
  \newcommand{\instrCenterY}{0}
  \newcommand{\instrMinX}{-\instrSize/2}
  \newcommand{\instrMinY}{-\instrSize/2}
  \newcommand{\instrMaxX}{\instrSize/2}
  \newcommand{\instrMaxY}{\instrSize/2}
  \newcommand{\instrCornerSize}{\instrSize*.1}

  \draw [color = green, line width=1pt] (\instrMinX, \instrMaxY-\instrCornerSize) --
                                                             (\instrMinX+\instrCornerSize, \instrMaxY) --
                                                             (\instrMaxX-\instrCornerSize, \instrMaxY) --
                                                             (\instrMaxX, \instrMaxY-\instrCornerSize) --
                                                             (\instrMaxX, \instrMinY+\instrCornerSize) --
                                                             (\instrMaxX-\instrCornerSize, \instrMinY) --
                                                             (\instrMinX+\instrCornerSize, \instrMinY) --
                                                             (\instrMinX, \instrMinY+\instrCornerSize) --
                                                             cycle;

%===== End of preambule ======



\draw [color=black] (\instrMinX * 0.8,\instrMinY * 0.8)  --(\instrMinX * 0.8,\instrMaxY * 0.3) ;
\draw [color=black] (\instrMinX * 0.8,\instrMinY * 0.8)  --(\instrMaxX * 0.8,\instrMinY * 0.8) ;

\draw (\instrMinX*.6, \instrMaxY*0.65) node{\textsf{\fbox{\large{08h05}}}};
\draw (\instrMaxX*.4, \instrMaxY*0.65) node{\textsf{\fbox{\large{14h27}}}};

\draw (\instrMinX*.6, \instrMaxY*0.4) node{\textsf{\fbox{\large{1.3 m}}}};
\draw (\instrMaxX*.4, \instrMaxY*0.4) node{\textsf{\fbox{\large{9.4 m}}}};



%\begin{tikzpicture}[domain=-2:2]
 % \draw[very thin,color=gray] (\instrMinX * 0.8,\instrMinY * 0.8, 2,1) grid (4,4);

 % \draw[color=blue]   plot (\x,{sin(\x r)}) ;   node[right] {$h$};

   %\draw[->] (-0.2,0) -- (4.2,0) node[right] {$x$};
  %\draw[->] (0,-1.2) -- (0,4.2) node[above] {$f(x)$};

%  \draw[color=red]    plot (\x,\x)             node[right] {$f(x) =x$};
  % \x r means to convert '\x' from degrees to _r_adians:
%  \draw[color=blue]   plot (\x,{sin(\x r)})    node[right] {$f(x) = \sin x$};
 % \draw[color=orange] plot (\x,{0.05*exp(\x)}) node[right] {$f(x) = \frac{1}{20} \mathrm e^x$};
%\end{tikzpicture}




%\draw [fill = brown, line width = 1pt] (\instrMinX * 0.8, \instrMinY*0.1) rectangle
%((\instrMinX * 0.2, \instrMinY*0.8) ;
%\draw [fill = cyan, line width = 1pt] (\instrMinX * 0.8, \instrMinY*0.1) rectangle
%((\instrMinX * 0.2, \instrMaxY*0.6) ;
%\draw (\instrMaxX*.4, 0) node{\textsf{\fbox{\Huge{4.3 m}}}};
%\draw [color = red](\instrMaxX*.4, \instrMaxY*0.5) node{\textsf{\fbox{\Huge{Depth}}}};

\end{tikzpicture}
