
%%\newglossaryentry{MEMS}{name=Nom, description={}}

\newglossaryentry{babord} {
	name=bâbord,
	sort={BABORD},
	description={désigne le côté gauche d'un navire, en tournant le dos à la poupe}
}

\newglossaryentry{tribord} {
	name=tribord,
	description={le côté droit d'un navire, en tournant le dos à la poupe}
}

\newglossaryentry{cavalement} {
	name=cavalement,
	sort={CAVALEMENT},
	description={Mouvement le long de l'axe longitudinal du navire, c.à.d. d'avant en arrière, ce mouvement correspond à une accélération ou un ralentissement.}
}

\newglossaryentry{embardee} {
	name=embardée,
	sort={EMBARDEE},
	description={Mouvement latéral du navire, c.à.d. vers son côté. Il est souvent provoqué par les rafales de vent et la houle. Il est souvent liée à la gîte et à la dérive}
}

\newglossaryentry{I2C} {
	name=I2C,
	description={Inter-Integrated Circuit en englais. C'est un bus de communication série à 2 fils à courte distance.}
}

\newglossaryentry{gite} {
	name=gîte,
	description={inclinaison latérale d'un navire. La gîte est généralement due au vent.}
}

\newglossaryentry{MEMS} {
    name=MEMS,
    description={Micro Electro Mechanicals Systems ou, en français, systèmes micro électromécaniques. Parmi ces systèmes figurent le magnétomètre, l'accéléromètre et le gyromètre.}
    }

\newglossaryentry{NMEA}
{
	name={NMEA}, % le terme à référencer (l'entrée qui apparaitra dans le glossaire)
	description={National Marine Electronics Association. Cet organisme américain a créé les protocoles NMEA~0183 et NMEA~2000}, % la description du terme(sans retour à la ligne)
	sort={NMEA}, % si le mot contient des caractère spéciaux, ils ne seront pas pris en compte
	plural={NMEA} % la forme plurielle du terme
}

\newglossaryentry{NMEA0183}
{
	name={NMEA0183}, % le terme à référencer (l'entrée qui apparaitra dans le glossaire)
	description={Protocole Normalisé et non officiellement documenté défini par la NMEA utilisé par des appareils de navigation}, % la description du terme(sans retour à la ligne)
	sort={NMEA0183}, % si le mot contient des caractère spéciaux, ils ne seront pas pris en compte
	plural={NMEA0183} % la forme plurielle du terme
}

\newglossaryentry{pilonnement} {
    name=pilonnement,
    sort={PILONNEMENT},
    description={Mouvement vertical du navire, c.à.d. de bas en haut et de haut en bas. ce mouvement est généralement provoqué par les vagues.}
    }

\newglossaryentry{roulis} {
	name=roulis,
	description={ Mouvement d'oscillation d'un bateau autour de l'axe longitudinal (en général sous l'effet de la houle ou du vent)}
}

\newglossaryentry{quaternion} {
	name=quaternion,
	sort={QUATERNION},
	description={Nombre hypercomplexe formé par une partie réelle et 3 parties imaginaires, il servent à représenter l'orientation du navire}
}

\newglossaryentry{lacet} {
	name=lacet,
	description={mouvement de rotation par rapport à l'axe vertical du navire}
    }

\newglossaryentry{tangage} {
	name=Tangage,
	description={mouvement de rotation par rapport à l'axe transversal du navire}
    }

\newglossaryentry{UART} {
	name=UART,
	description={Universal Asynchronous Receiver Transmitter soit émetteur-récepteur asynchrone universel, aussi appelée liaison série.}
}
